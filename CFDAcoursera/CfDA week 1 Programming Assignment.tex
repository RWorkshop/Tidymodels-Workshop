\documentclass[a4paper,12pt]{article}
%%%%%%%%%%%%%%%%%%%%%%%%%%%%%%%%%%%%%%%%%%%%%%%%%%%%%%%%%%%%%%%%%%%%%%%%%%%%%%%%%%%%%%%%%%%%%%%%%%%%%%%%%%%%%%%%%%%%%%%%%%%%%%%%%%%%%%%%%%%%%%%%%%%%%%%%%%%%%%%%%%%%%%%%%%%%%%%%%%%%%%%%%%%%%%%%%%%%%%%%%%%%%%%%%%%%%%%%%%%%%%%%%%%%%%%%%%%%%%%%%%%%%%%%%%%%
\usepackage{eurosym}
\usepackage{vmargin}
\usepackage{amsmath}
\usepackage{graphics}
\usepackage{epsfig}
\usepackage{subfigure}
\usepackage{fancyhdr}
%\usepackage{listings}
\usepackage{framed}
\usepackage{graphicx}

\setcounter{MaxMatrixCols}{10}
%TCIDATA{OutputFilter=LATEX.DLL}
%TCIDATA{Version=5.00.0.2570}
%TCIDATA{<META NAME="SaveForMode" CONTENT="1">}
%TCIDATA{LastRevised=Wednesday, February 23, 2011 13:24:34}
%TCIDATA{<META NAME="GraphicsSave" CONTENT="32">}
%TCIDATA{Language=American English}

\pagestyle{fancy}
\setmarginsrb{20mm}{0mm}{20mm}{25mm}{12mm}{11mm}{0mm}{11mm}
\lhead{Dublin \texttt{R}} \rhead{Week 1}
\chead{Computing For Data Analysis }
%\input{tcilatex}


% http://www.norusis.com/pdf/SPC_v13.pdf
\begin{document}

% 1 - names - ready
% 2 - head - ready
% 3 - dim() -ready
% 4 - tail -ready
% 5 - Square Brackets -ready
% 6 - Missing Values - summary?
% 7 - summary (missing value remove) 
% 8 - subset + relational and logical operators
% 9 - str() - ready
%10 - subset according to category
 
\section{Reading in a CSV file}
% Section 1
To read in a csv file, it is convenient to save it to your \textit{\textbf{working directory}}, the default file directory that \texttt{R} uses. Once it is in the working directory, you can use the \texttt{read.csv()} command to load it into the \texttt{R} environment.
\begin{framed}
\begin{verbatim}
getwd()
#[1] "C:/Users/Computer5/Documents"

HW0 <- read.csv("hw0_data.csv",header = TRUE)
\end{verbatim}
\end{framed}
You can change the working directory using the \texttt{setwd()} to the one you require.
\begin{framed}
\begin{verbatim}
getwd()
#[1] "C:/Users/Computer5/Documents"

setwd("C:/Users/Computer5/Documents/R")

#  > getwd()
#  [1] "C:/Users/Computer5/Documents/R"
\end{verbatim}
\end{framed}




