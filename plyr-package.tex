\documentclass[12pt]{article}
\usepackage{amsmath}
\usepackage{amssymb}
\usepackage{framed}
\usepackage{graphicx}

\begin{document}

\section{\texttt{plyr}: }

\texttt{plyr} is a set of tools for a common set of problems: you need to split up a big data structure into homogeneous pieces, apply a function to each piece and then combine all the results back together. For example, you might want to:

\begin{itemize}
\item fit the same model to subsets of a data frame
\item quickly calculate summary statistics for each group
\item perform group-wise transformations like scaling or standardising
\end{itemize}
It’s already possible to do this with base \texttt{R} functions (like split and the apply family of functions), but \texttt{plyr} makes it all a bit easier with:

\begin{itemize}
\item totally consistent names, arguments and outputs
\item convenient parallelisation through the foreach package
\item input from and output to data.frames, matrices and lists
\item progress bars to keep track of long running operations
\item built-in error recovery, and informative error messages
\item labels that are maintained across all transformations
\end{itemize}

\end{document}