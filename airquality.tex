# Exercise: The `airquality` Data Set

The `airquality` dataset in R contains daily readings of air quality measurements in New York from May to September 1973. It consists of **154 observations** across **6 variables**:

### Variables

| Variable  | Type     | Description                       |
|-----------|----------|-----------------------------------|
| Ozone     | numeric  | Ozone (ppb)                       |
| Solar.R   | numeric  | Solar Radiation (lang)            |
| Wind      | numeric  | Wind Speed (mph)                  |
| Temp      | numeric  | Temperature (°F)                  |
| Month     | numeric  | Month (1 = Jan, ..., 12 = Dec)    |
| Day       | numeric  | Day of the month (1–31)           |

### Initial Exploration

```r
tail(airquality)
help(airquality)
```

---

## Exercises

1. For each variable, how many **missing values** are there?
2. How many **complete cases** are in the dataset?
3. What is the **variance** of each continuous variable?
4. How many complete cases are there **excluding missing values**?
5. If you ignore the `Wind` variable, how many **partially complete** cases remain?

---

## Working with Complete Cases

The function `complete.cases()` helps identify rows without any missing values. It returns a logical vector of length equal to the number of rows in the dataset.

You can use `as.numeric()` to convert logicals into 1s (`TRUE`) and 0s (`FALSE`):

```r
X <- c(TRUE, TRUE, FALSE, FALSE, TRUE)
as.numeric(X)
# [1] 1 1 0 0 1
```

This technique is handy for counting complete or incomplete observations in your data.

---
